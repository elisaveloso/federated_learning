% /home/elisaveloso/federated_learning/chapters/chapters/introduction.tex
% Introduction chapter for a thesis/report. Insert into main document using \include or \input.

\chapter{Introduction}
\label{chap:introduction}

\begin{abstract}
This chapter introduces the problem domain, motivates the work, summarizes the main contributions,
and outlines the structure of the thesis. Replace the placeholders below with concise text
that reflects your project.
\end{abstract}

\section{Motivation}
% Explain why the topic is important, relevant applications, and high-level challenges.
Federated learning enables collaborative model training across multiple devices or institutions
while keeping data localized. This approach addresses privacy, communication, and data
heterogeneity concerns that arise in centralized learning systems.

\section{Background and problem statement}
% Brief background and a precise statement of the research problem.
Provide a short technical background on machine learning, distributed optimization, and privacy-preserving techniques.
State the concrete problem you address, including assumptions and scope.

Example: consider minimizing a global objective aggregated from K clients,
\[
    \min_{w\in\mathbb{R}^d} F(w) \;=\; \sum_{k=1}^K p_k F_k(w),
\]
where \(F_k\) is the local objective at client \(k\) and \(p_k\ge 0\), \(\sum_k p_k = 1\).
Discuss challenges such as communication efficiency, non-iid data, and system heterogeneity.

\section{Contributions}
List the main contributions of the thesis/report. Use bullet points for clarity.
\begin{itemize}
    \item A novel algorithm for communication-efficient federated learning that handles non-iid local data.
    \item Theoretical convergence guarantees under realistic client sampling and delay models.
    \item Empirical evaluation on standard benchmarks demonstrating improved accuracy and reduced communication.
\end{itemize}

\section{Methodology (brief)}
Summarize the approach and methods used: algorithm design, theoretical analysis, and experiments.
Mention datasets, baselines, and evaluation metrics you plan to use.

\section{Organization of the thesis}
Outline the structure of the remaining chapters.
\begin{itemize}
    \item Chapter~\ref{chap:related} reviews related work and places this research in context.
    \item Chapter~\ref{chap:method} describes the proposed algorithm and theoretical analysis.
    \item Chapter~\ref{chap:experiments} presents experimental setup and results.
    \item Chapter~\ref{chap:conclusion} concludes and discusses future directions.
\end{itemize}

\section{Notation}
Introduce commonly used notation for the rest of the document.
\begin{itemize}
    \item \(K\): number of clients.
    \item \(n_k\): number of samples at client \(k\).
    \item \(w\in\mathbb{R}^d\): global model parameters.
    \item \(\nabla F_k(w)\): gradient of local objective at client \(k\).
\end{itemize}

% TODO: replace placeholders with detailed content, add citations (\cite{author:year}), figures, and tables as needed.